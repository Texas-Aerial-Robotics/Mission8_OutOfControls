The below functions will help you to easily create highlevel path planning programs. This A\+PI is designed to be used with the Ardu\+Copter flight stack. \mbox{\hyperlink{group__control__functions}{Control\+\_\+functions}}

The below code is a simple example that executes a square flight pattern. \mbox{\hyperlink{controlAPIExample_8cpp}{src/control\+A\+P\+I\+Example.\+cpp}}


\begin{DoxyCode}{0}
\DoxyCodeLine{\#include <control\_functions.hpp>}
\DoxyCodeLine{//include API }
\DoxyCodeLine{}
\DoxyCodeLine{int main(int argc, char** argv)}
\DoxyCodeLine{\{}
\DoxyCodeLine{    //initialize ros }
\DoxyCodeLine{    ros::init(argc, argv, "offb\_node");}
\DoxyCodeLine{    ros::NodeHandle controlnode;}
\DoxyCodeLine{    }
\DoxyCodeLine{    //initialize control publisher/subscribers}
\DoxyCodeLine{    init\_publisher\_subscriber(controlnode);}
\DoxyCodeLine{}
\DoxyCodeLine{    //specify some waypoints }
\DoxyCodeLine{    std::vector<control\_api\_waypoint> waypointList;}
\DoxyCodeLine{    control\_api\_waypoint nextWayPoint;}
\DoxyCodeLine{    nextWayPoint.x = 0;}
\DoxyCodeLine{    nextWayPoint.y = 0;}
\DoxyCodeLine{    nextWayPoint.z = 3;}
\DoxyCodeLine{    nextWayPoint.psi = 0;}
\DoxyCodeLine{    waypointList.push\_back(nextWayPoint);}
\DoxyCodeLine{    nextWayPoint.x = 5;}
\DoxyCodeLine{    nextWayPoint.y = 0;}
\DoxyCodeLine{    nextWayPoint.z = 3;}
\DoxyCodeLine{    nextWayPoint.psi = -90;}
\DoxyCodeLine{    waypointList.push\_back(nextWayPoint);}
\DoxyCodeLine{    nextWayPoint.x = 5;}
\DoxyCodeLine{    nextWayPoint.y = 5;}
\DoxyCodeLine{    nextWayPoint.z = 3;}
\DoxyCodeLine{    nextWayPoint.psi = 0;}
\DoxyCodeLine{    waypointList.push\_back(nextWayPoint);}
\DoxyCodeLine{    nextWayPoint.x = 0;}
\DoxyCodeLine{    nextWayPoint.y = 5;}
\DoxyCodeLine{    nextWayPoint.z = 3;}
\DoxyCodeLine{    nextWayPoint.psi = 90;}
\DoxyCodeLine{    waypointList.push\_back(nextWayPoint);}
\DoxyCodeLine{    nextWayPoint.x = 0;}
\DoxyCodeLine{    nextWayPoint.y = 0;}
\DoxyCodeLine{    nextWayPoint.z = 3;}
\DoxyCodeLine{    nextWayPoint.psi = 180;}
\DoxyCodeLine{    waypointList.push\_back(nextWayPoint);}
\DoxyCodeLine{}
\DoxyCodeLine{    // wait for FCU connection}
\DoxyCodeLine{    wait4connect();}
\DoxyCodeLine{}
\DoxyCodeLine{    //wait for used to switch to mode GUIDED}
\DoxyCodeLine{    wait4start();}
\DoxyCodeLine{}
\DoxyCodeLine{    //create local reference frame }
\DoxyCodeLine{    initialize\_local\_frame();}
\DoxyCodeLine{}
\DoxyCodeLine{    //request takeoff}
\DoxyCodeLine{    takeoff(3);}
\DoxyCodeLine{}
\DoxyCodeLine{    //specify control loop rate. We recommend a low frequency to not over load the FCU with messages. Too many messages will cause the drone to be sluggish}
\DoxyCodeLine{    ros::Rate rate(2.0);}
\DoxyCodeLine{    int counter = 0;}
\DoxyCodeLine{    while(ros::ok())}
\DoxyCodeLine{    \{}
\DoxyCodeLine{        ros::spinOnce();}
\DoxyCodeLine{        rate.sleep();}
\DoxyCodeLine{        if(check\_waypoint\_reached() == 1)}
\DoxyCodeLine{        \{}
\DoxyCodeLine{            if (counter < waypointList.size())}
\DoxyCodeLine{            \{}
\DoxyCodeLine{                set\_destination(waypointList[counter].x,waypointList[counter].y,waypointList[counter].z, waypointList[counter].psi);}
\DoxyCodeLine{                counter++;  }
\DoxyCodeLine{            \}else\{}
\DoxyCodeLine{                //land after all waypoints are reached}
\DoxyCodeLine{                land();}
\DoxyCodeLine{            \}   }
\DoxyCodeLine{        \}   }
\DoxyCodeLine{        }
\DoxyCodeLine{    \}}
\DoxyCodeLine{    return 0;}
\DoxyCodeLine{\}}
\end{DoxyCode}
 \subsection*{run example code }


\begin{DoxyCode}{0}
\DoxyCodeLine{roslaunch mission8\_sim droneOnly.launch}
\DoxyCodeLine{\# New Terminal}
\DoxyCodeLine{~/catkin\_ws/src/Mission8\_OutOfControls/scripts/startsim.sh}
\DoxyCodeLine{\# New Terminal}
\DoxyCodeLine{roslaunch out\_of\_controls apm.launch}
\DoxyCodeLine{\# New Terminal }
\DoxyCodeLine{roslaunch out\_of\_controls contolAPIExample.launch}
\end{DoxyCode}
 N\+O\+T\+E$\ast$$\ast$ you can tile gnome terminals by pressing {\ttfamily ctrl + shift + t} 